% custom commands
\renewcommand{\vec}[1]{\bm{#1}}

% numbered dividers depth & TOC depth
\setsecnumdepth{subsection}
\settocdepth{subsection}

% set dividers style (font and size)

% chapters head
\renewcommand{\chapnumfont}{\Huge\bfseries\alt}
\renewcommand{\chaptitlefont}{\Huge\bfseries\alt}

\renewcommand{\printchaptername}{} % remove "Capitolo" in chapter title
\renewcommand{\printchapternum}{} % remove chapter num in chapter title

% remove page numbers from chapter pages (using empty pagestyle)
\let\oldchapter\chapter
\renewcommand{\chapter}[1]{
    \oldchapter{#1}
    \thispagestyle{empty}
}

% chapter title style
\renewcommand{\printchaptertitle}[1]{
    \centering\chaptitlefont
    \ifnum\value{chapter} > 0
        \raisebox{15pt}{\scalebox{.4}{\pgfornament{11}}}\hspace{5pt}
        {\fontsize{50pt}{0pt}\alt\thechapter}
        \hspace{5pt}\raisebox{15pt}{\scalebox{.4}{\pgfornament{14}}}
        \\[10pt]
        \hrule height 2pt \vspace{2pt} \hrule height 1pt \vspace{10pt}
        #1
        \vspace{10pt} \hrule height 1pt \vspace{2pt} \hrule height 2pt
    \else
        \raisebox{15pt}{\scalebox{.45}{\pgfornament{46}}}\\[10pt]
        {\fontsize{40pt}{0pt}\alt #1}\\
        \raisebox{15pt}{\scalebox{.6}{\pgfornament{88}}}\\
    \fi
}

% define custom font size
\makeatletter
\newcommand{\fortypt}{\@setfontsize\fortypt{40}{0}} % Adjust the numbers as needed
\newcommand{\twentytwopt}{\@setfontsize\twentytwopt{22}{0}}
\makeatother

% sections head
\setsecheadstyle{\Large\bfseries\alt}
\setsubsecheadstyle{\large\bfseries\alt}
\setsubsubsecheadstyle{\bfseries\alt}

% abstract config
\addto{\captionsitalian}{\renewcommand{\abstractname}{
        \raisebox{15pt}{\scalebox{.5}{\pgfornament{45}}}\\[10pt]
        Abstract\\[15pt]
        \raisebox{15pt}{\scalebox{.6}{\pgfornament{88}}}\\
}}

% image path config
\graphicspath{{assets/figures/}}

% bibliography config
\addbibresource{include/back/refs.bib}
\defbibheading{bibliography}[\refname]{
    \centering
    \vspace*{1cm}
    \alt\bfseries\fortypt Bibliografia \\[15pt]
    \raisebox{15pt}{\scalebox{.7}{\pgfornament{88}}}\\
    \vspace{.5cm}
}

% hyperref setup
\definecolor{indigo}{HTML}{3085ac}
\definecolor{unipd}{HTML}{9B0014}
\definecolor{grayfiller}{RGB}{245, 245, 245}
\definecolor{grayborder}{RGB}{230, 230, 230}

\hypersetup{
    colorlinks = true,
    linkbordercolor = white,
    linkcolor = unipd,
    urlcolor = indigo,
    citecolor = unipd
}

% change the url font
\renewcommand{\UrlFont}{\alt}

% fix compile warning PDF string (hyperref)
\pdfstringdefDisableCommands{
    \def\alt{}
}

% code snippets config
\newcommand{\algtitle}[3]{\alt
    {\bfseries Algoritmo}: {\jbm #1} \\
    {\bfseries Input}: #2 \\
    {\bfseries Output}: #3
}

\newmdenv[style=listing]{internalcode}

\newenvironment{code}[2]{
    \begin{internalcode}[
        frametitle=#1,
        frametitlealignment=#2
     ]
}
{
    \end{internalcode}
}

\newenvironment{captioncode}[1]{
    \captionof{lstlisting}{#1}
    \begin{internalcode}
}
{
    \end{internalcode}
}

\newenvironment{wrapper}{}{}

\mdfdefinestyle{listing}{
   backgroundcolor = grayfiller,
   linecolor = grayborder,
   linewidth = 2pt,
   frametitlefont = \alt,
   frametitlealignment = \centering,
   frametitlebelowskip = 5pt,
   frametitlerule = true,
   frametitlerulewidth = 1.5pt,
}

\lstdefinestyle{default}{
    basicstyle = \jbm,
    tabsize = 4,
    numbers = left,
    numberstyle = \small\jbm,
    numbersep = 5pt,
    numberblanklines = false,
    columns = flexible,
    breaklines = true,
    captionpos = t,
    abovecaptionskip = -5pt,
    belowcaptionskip = 5pt,
    keywordstyle = \bfseries\color{indigo},
    commentstyle = \itshape\color[HTML]{157F1F},
    stringstyle = \color[HTML]{F45B69},
    showstringspaces = false,
    escapechar = §,
    xleftmargin = 15pt,
    breakatwhitespace = true,
    breakautoindent = true,
}

% theorems config

\declaretheoremstyle[
    numberwithin=chapter,
    headfont={\color{darkgray}\large\fontseries{sb}\alt},
    notefont={\fontseries{mid}\alt},
    postheadspace=5pt,
    spaceabove=10pt,
    spacebelow=10pt,
]{defstyle}

\declaretheorem[
    style=defstyle,
    name=Definizione,
    refname={definizione,definizioni},
    Refname={Definizione,Definizioni},
]{definition}

\declaretheorem[
    style=defstyle,
    name=Teorema,
    refname={teorema,teoremi},
    Refname={Teorema,Teoremi}
]{theorem}

\declaretheorem[
    style=defstyle,
    name={Teorema Fondamentale della Programmazione Lineare},
    numbered=no
]{LPTheorem}

\mdfdefinestyle{definition}{
    linewidth=2pt,
    topline=false,
    bottomline=false,
    rightline=true,
    leftline=true,
    innertopmargin=-5pt,
    linecolor=grayborder,
    backgroundcolor=grayfiller,
}

\mdfdefinestyle{theorem}{
    linewidth=2pt,
    topline=true,
    bottomline=true,
    rightline=false,
    leftline=false,
    innertopmargin=0pt,
    innerbottommargin=10pt,
    linecolor=grayborder,
    backgroundcolor=grayfiller,
}

\surroundwithmdframed[style=definition]{definition}
\surroundwithmdframed[style=theorem]{theorem}
\surroundwithmdframed[style=theorem]{LPTheorem}

\makeatletter
\newenvironment{breakablealgorithm}
  {% \begin{breakablealgorithm}
   \begin{center}
     \refstepcounter{algorithm}% New algorithm
     \hrule height.8pt depth0pt \kern2pt% \@fs@pre for \@fs@ruled
     \renewcommand{\caption}[2][\relax]{% Make a new \caption
       {\raggedright\textbf{\fname@algorithm~\thealgorithm} ##2\par}%
       \ifx\relax##1\relax % #1 is \relax
         \addcontentsline{loa}{algorithm}{\protect\numberline{\thealgorithm}##2}%
       \else % #1 is not \relax
         \addcontentsline{loa}{algorithm}{\protect\numberline{\thealgorithm}##1}%
       \fi
       \kern2pt\hrule\kern2pt
     }
  }{% \end{breakablealgorithm}
     \kern2pt\hrule\relax% \@fs@post for \@fs@ruled
   \end{center}
  }
\makeatother