% abstract config
\renewcommand{\abstractnamefont}{\fortypt\bfseries\alt}

\cleardoublepage
\thispagestyle{empty}
\begin{abstract}
% <<< ABSTRACT <<<<<<<<<<<<<<<<<<<<<<<<<<<<<<<<<<<<<<<<<<<<<<<<<<<<<<<<<<<<
\noindent Questa tesi si propone di analizzare e confrontare alcuni algoritmi crittografici resistenti agli attacchi dei computer quantistici, noti come algoritmi post-quantum, recentemente valutati dal NIST (National Institute of Standards and Technology). L'obiettivo principale è valutare le performance di alcune implementazioni esistenti di algoritmi di firma digitale e di generazione delle chiavi in termini di sicurezza, efficienza e scalabilità. La tesi include una revisione della letteratura sugli algoritmi post-quantum, un'analisi comparativa delle soluzioni proposte, e una valutazione delle loro potenziali applicazioni in scenari reali. Questo lavoro mira a contribuire alla comprensione e alla valutazione delle tecnologie post-quantum emergenti, offrendo una base per future implementazioni sicure nell'ambito della protezione delle identità digitali.
\vspace{0.4cm}
\begin{center}
   \line(1,0){360}
\end{center}
\vspace{0.5cm}
\noindent This thesis aims to analyze and compare certain cryptographic algorithms resistant to attacks by quantum computers, known as post-quantum algorithms, recently evaluated by NIST (National Institute of Standards and Technology). The primary objective is to assess the performance of some existing implementations of digital signature algorithms and key generation in terms of security, efficiency, and scalability. The thesis includes a literature review on post-quantum algorithms, a comparative analysis of the proposed solutions, and an evaluation of their potential applications in real-world scenarios. This work aims to contribute to the understanding and assessment of emerging post-quantum technologies, providing a basis for future secure implementations in the field of digital identity protection.


% >>>>>>>>>>>>>>>>>>>>>>>>>>>>>>>>>>>>>>>>>>>>>>>>>>>>>>>>>>>>>>>>>>>>>>>>>
\end{abstract}
\cleardoublepage

