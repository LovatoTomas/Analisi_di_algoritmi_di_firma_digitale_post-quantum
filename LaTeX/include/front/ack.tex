\cleardoublepage
\thispagestyle{empty}
\begin{center}
%    \raisebox{15pt}{\scalebox{1.5}{\pgfornament{113}}}\\[15pt]
    \fortypt\bfseries\alt{Ringraziamenti}\\[15pt]
    \raisebox{15pt}{\scalebox{.7}{\pgfornament{88}}}\\
\end{center}

% <<< RINGRAZIAMENTI <<<<<<<<<<<<<<<<<<<<<<<<<<<<<<<<<<<<<<<<<<<<<<<<<<<<<<

Desidero innanzitutto ringraziare il Professor Luca Boldrin per la sua preziosa guida nella stesura di questa tesi. La sua competenza e disponibilità mi hanno permesso di esplorare un settore affascinante e di raggiungere risultati che non avrei mai immaginato. Il suo suggerimento di approfondire gli algoritmi post-quantum mi ha aperto a nuove interessanti prospettive.

Un ringraziamento speciale va alla mia famiglia, che mi ha sempre sostenuto e incoraggiato a perseguire i miei obiettivi. La loro fiducia incondizionata nei miei confronti e nelle mie scelte si è trasformata in una forza indispensabile, soprattutto nei momenti in cui mi sentivo sopraffatto dagli esami o dalle mille difficoltà della vita. Un caloroso grazie a Roberta, Claudio e Sara.

Come non ringraziare, poi, tutte le persone che da tanti anni condividono con me gioie e difficoltà, accompagnandomi lungo i vari percorsi della vita. Ogni momento trascorso insieme è un tesoro inestimabile. Un grazie di cuore ad Alberto, Francesco, Matteo e Mattia. Spesso, tra un’attività e l’altra, mi capita di fermarmi e perdermi nei ricordi dei momenti più divertenti che abbiamo vissuto insieme... non considero questi istanti una distrazione, ma piuttosto un simbolo del prezioso legame che ci unisce.

In questa pagina non può mancare un \textit{team} che si è rivelato importante nella mia vita: ispiro.tech. Da più di due anni faccio parte di questa realtà, resa speciale e stimolante dalle persone che ne fanno parte: Davide, Simone e Mattia.

Infine, vorrei ringraziare tutte le persone che ho avuto il piacere di conoscere più a fondo solo di recente. In particolare, un grande grazie a Tommaso, Leonardo, Hermann, Giulia, Greta, Filippo ed Enrico. Spero di poter approfondire il nostro rapporto, così come con tutti coloro che, per ragioni di spazio, non ho potuto menzionare.

% >>>>>>>>>>>>>>>>>>>>>>>>>>>>>>>>>>>>>>>>>>>>>>>>>>>>>>>>>>>>>>>>>>>>>>>>>

\begin{flushright}
    \textsc{La macchina può calcolare, ma è il cuore umano che dà valore ai numeri.}
\end{flushright}
\vspace{.5cm}
%\begin{center}
%    \raisebox{15pt}{\scalebox{.8}{\pgfornament{87}}}\\[10pt]
%\end{center}
\cleardoublepage

